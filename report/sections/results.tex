% !TEX root=../main.tex
\section{Results}
\label{sec:results}

We were pleased to be able to succesffully implement autonomous control of a
TurtleBot using an end-to-end visual control approach. The classical approach served as a meaningful learning experience and also a useful tool in collecting data to train the neural network approach, which was our end goal.

\begin{figure}[hbt]
  \includegraphics[width=\columnwidth]{figures/success_track}
  \caption{Snapshot of the TurtleBot successfully following a track using end-to-end deep learning-based control.}
  \label{fig:success_track}
\end{figure}

Some of the main advantages we saw of using the neural network control as opposed to the classical method was the ability to implicitly encode variability and robustness into the control simply by training on more data in varied environments. For example, the classical method, which depended heavily on proper segmentation, needed to be re-tuned for each ground surface and lighting condition, whereas the neural network was able to learn how to extract the proper features in multiple environments without the need to explicitly retune or revise the network.

It was also interesting to note the information the neural network was able to
learn. To visualize the salient features in a given image, we evaluated the
gradient of the output of the network with respect to the input image at the
given image. We then normalized these gradeints to turn them into a visible
grayscale image, equal in size to that of the input image. As seen in
Figures~\ref{fig:saliency1}~through~\ref{fig:saliency3}, the brightest pixels in
the gradient images correspond to the pixels that most contributed to the output
of the neural network. For the most part, these salient pixels correspond to the areas where the rope is seen in the input image. Note how the neural network learned to identify the rope specifically, not just white objects, as the light reflection in Figure~\ref{fig:saliency3} is filtered out and does not contribute much to the output. These glare spots were difficult for the classical method to handle, which conveys a clear advantage of the network-based control over the classical control.

\begin{figure}[hbt]
  \includegraphics[width=\columnwidth]{figures/saliency1}
  \caption{Top: Original image; bottom: Class saliency for rope on carpet data. Shows the gradient is highest near where the rope appears in the original image (top)}
  \label{fig:saliency1}
\end{figure}

\begin{figure}[hbt]
  \includegraphics[width=\columnwidth]{figures/saliency2}
  \caption{Class saliency on reflective concrete flooring.}
  \label{fig:saliency2}
\end{figure}

\begin{figure}[hbt]
  \includegraphics[width=\columnwidth]{figures/saliency3}
  \caption{Class saliency on reflecitve concrete floor with mulptiple glare spots. Show that the network learned to filter out bright contours that do not correspond with rope.}
  \label{fig:saliency3}
\end{figure}
